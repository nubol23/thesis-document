\section*{DEDICATORIA}
\addcontentsline{toc}{section}{DEDICATORIA}
\vfill
\hfill\begin{minipage}{\dimexpr\textwidth-9cm}
Dedico la presente tesis a mi familia por todo el apoyo, amor y paciencia durante todos estos años de estudio.
\end{minipage}

\newpage
\section*{\hspace{5.5cm}AGRADECIMIENTOS}
\addcontentsline{toc}{section}{AGRADECIMIENTOS}

\begin{center}
	A mi familia por el apoyo durante todos los años cursando la carrera.\\
	\vspace{0.2cm}
	A mi tutor M.Sc. Aldo Valdez Alvarado por la guía durante el desarrollo de la presente tesis, y a mi asesora Lic. Brigida Carvajal Blanco por los consejos, sugerencias y apoyo.
\end{center}

\blfootnote{e-mail: rafael.rvp98@gmail.com}

\newpage
\section*{\hspace{6.4cm}RESUMEN}
\addcontentsline{toc}{section}{RESUMEN}
La creciente popularidad de vehículos de distintas marcas con funcionalidades autónomas, ocasionó la creación de simuladores y conjuntos de datos para el entrenamiento de modelos de conducción mediante técnicas de visión artificial y aprendizaje profundo.

El problema a abordar en las soluciones que se desarrollan es crear implementaciones eficientes para realizar inferencias rápidas y reaccionar a las distintas situaciones ambientales mediante cámaras.

Se propone un modelo, desarrollado aplicando la metodología CRISP-DM, compuesto por redes neuronales y algoritmos de visión computacional que se complementen, y así obtener una conducción autónoma básica y eficiente computacionalmente.\\

\noindent Palabras clave: Aprendizaje profundo, Visión computacional, Redes neuronales, Carla simulator, CRISP-DM

\newpage
\section*{\hspace{6.4cm}ABSTRACT}
\addcontentsline{toc}{section}{ABSTRACT}
The increasing popularity of different car manufacturers with autonomous features, made possible the creation of simulators and data sets to train self driving models using Computer Vision techniques and Deep Learning.

The main focus of these solutions is to develop efficient implementations to perform fast inference and react to different environmental situations through cameras.

In this work a model composed of deep neural networks and computer vision algorithms which work together is proposed, developed using CRISP-DM methodology,in order to obtain an efficient and basic self driving capability.\\

\noindent Keywords: Deep learning, Computer vision, Neural networks, Carla simulator, CRISP-DM