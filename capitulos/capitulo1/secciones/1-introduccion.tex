\section{INTRODUCCIÓN}
%\section[Introducción]{INTRODUCCIÓN}
Los vehículos autónomos son algo común en el imaginario colectivo, esto debido a sus apariciones en la ciencia ficción y a las noticias de que empresas como Waymo, Uber y muchas más están trabajando para lograr la tan deseada autonomía completa.

Los modelos de Machine Learning usados en la actualidad fueron desarrollados muchas décadas atrás, estos requerían pre procesar demasiado la información de conjuntos de datos pequeños, sin embargo gracias a los avances en la transmisión de la información y al acceso a datos masivos que antes no eran posibles de almacenar, estos requisitos fueron decreciendo mientras más datos de entrenamiento se tenían disponibles, es así como se da un resurgimiento en el interés por los modelos probabilísticos, superando estos a los basados en reglas, utilizados comúnmente hasta los años 90. \citep{Goodfellow-et-al-2016}
% (Goodfellow, Bengio, \& Courville, 2015)

En el ámbito de la visión computacional, los primeros modelos eran capaces de detectar objetos simples en imágenes muy pequeñas \citep{Rumelhart_Hinton_Williams_1986}. Por contraparte los modelos modernos basados en Redes Neuronales Convolucionales, son capaces de detectar por lo menos 1000 categorías distintas con una exactitud muy superior a los métodos clásicos siempre y cuando la cantidad de datos sea suficiente. \citep{alexnet}
% (Krizhevsky, Sutskever, \& Hinton, 2012)\\

Aprovechando los avances de la visión computacional y la mejora de mejores sensores, los sistemas de conducción autónoma refinaron su capacidad de percepción, etapa clave para la toma de decisiones, mediante el uso de las Redes Neuronales para la detección, segmentación y clasificación de objetos en las carreteras, con los más avanzados como el autopilot de Tesla detectando situaciones adversas basado en acciones de otros vehículos en el camino a tiempo real. \citep{karpathy-scaledml}
% (Karpathy, 2020)

En el presente trabajo se propone un modelo compuesto por redes neuronales convolucionales y algoritmos de visión artificial para resolver las tareas básicas requeridas para la percepción y control de un vehículo autónomo.
\newpage