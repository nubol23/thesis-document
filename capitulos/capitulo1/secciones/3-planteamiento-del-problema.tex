\section{PLANTEAMIENTO DEL PROBLEMA}
%\section[Plantemaiento del Problema]{PLANTEAMIENTO DEL PROBLEMA}
Si bien la idea de construir vehículos autónomos no es nueva, es en esta época donde finalmente se está logrando salir de etapas prototipo, a incluir estos sistemas en vehículos de venta masiva, por parte de distintos fabricantes, de entre los cuales uno de los más reconocidos es Tesla Motors, que a diferencia de las alternativas propuestas por Google y Uber, proponen un sistema sin el uso de sensores costosos y poco estéticos para un vehículo comercial como un LIDAR \citep{granath_2020}, basandose sólamente en sensores de proximidad, radar, GPS y múltiples cámaras, diseñando y entrenando sus modelos de aprendizaje profundo como arquitecturas de redes neuronales y aprendizaje reforzado, con los datos masivos recopilados de los vehículos vendidos que circulan día a día \citep{karpathy-scaledml}, buscando que sus vehículos sean capaces de conducirse en todo tipo de situaciones climáticas y de iluminación, ya que los modelos de aprendizaje profundo superan a los algoritmos de visión computacional en estas situaciones \citep{Goodfellow-et-al-2016}, si se tiene la gran cantidad de datos para entrenarlos correctamente. \citep{alexnet}
% , usando estas últimas para las tareas de percepción y toma de decisiones inmediatas aplicando distintos algoritmos de visión computacional clásica y moderna con Deep Learning.

Un coche que se conduzca solo, respetando todas las leyes de tránsito, con tiempos de reacción instantáneos y sin problemas humanos como el cansancio, en un ecosistema compuesto en su totalidad por vehículos autónomos, reduciría el riesgo de accidentes de tránsito. Existe un historial de accidentes en los que están envueltos autos autónomos, de estos, la mayoría fueron causados por otros conductores humanos que al no respetar las reglas de tránsito impactaron de alguna manera con el vehículo autónomo \citep{wikipedia_2020}. Por esta razón y debido a la forma de vida acelerada en la sociedad actual, automatizaciones que ahorren tiempo en alguna actividad, son productos atractivos en los que se invierte para su investigación, existiendo en el caso de los vehículos autónomos datasets como el Open Waymo dataset, el cual contiene 2TB de datos para que los concursantes implementen sistemas de conducción autónoma y compitan en distintos retos. \citep{waymo}
% Por esta razón y porque en estos tiempos dónde las personas están cada vez más ocupadas con sus obligaciones y actividades, cualquier tarea que sea considerada un gasto de tiempo y sea automatizable, siempre será un factor atractivo. Uno en el que las empresas automotrices están invirtiendo y debido a esto existe un gran interés científico por investigar métodos eficientes para llevarlo a cabo.

Aparte de aplicaciones en la movilidad doméstica, existen áreas de investigación en vehículos autónomos para competencias tanto a escala como con vehículos reales, y también se busca aplicar esta tecnología en situaciones de trabajo con maquinaria pesada que conlleva cierto riesgo en la operación de esta.

\subsection{PROBLEMA CENTRAL}
%\subsection[Problema Central]{PROBLEMA CENTRAL}
% ¿De qué manera se puede conseguir un nivel de autonomía básico en la conducción de un vehículo en pistas y carreteras usando solamente cámaras y algoritmos de visión computacional y modelos aprendizaje profundo?
%¿De qué manera conseguir un nivel de autonomía básico en la conducción de un vehículo en pistas y carreteras?
¿De qué manera conseguir un nivel básico de conducción autónoma en vías de doble sentido?

\subsection{PROBLEMAS SECUNDARIOS}
%\subsection[Problemas Secundarios]{PROBLEMAS SECUNDARIOS}
\begin{itemize}[nosep]
%    \item Los conjuntos de datos disponibles contienen información de distintos sensores y cámaras sin procesamiento previo, debido a esto los datos pesan demasiado y existe mucha variabilidad en la distribución de imágenes.
%    \item Para lograr una buena autonomía son necesarios componentes de hardware avanzados como un LIDAR que brindan una mejor percepción del espacio, esto ocasiona que el precio de los sistemas de conducción autónoma sea elevado, poco accesible a estos componentes y complejiza el problema de implementación.
%    \item Los algoritmos de visión computacional requieren poca capacidad de cómputo, pero sólo funcionan en condiciones ideales de imágenes para los que se programan, esto ocasiona problemas con cambios en el contenido, perspectiva e iluminación.
%    \item Los modelos de aprendizaje profundo generalizan bien pero un modelo de varias capas es más complejo de entrenar, hace necesario recopilar muchos datos y requiere mucho poder de cómputo.
%    \item Los sistemas basados en reglas de decisión predefinidas no logran extrapolar el conocimiento correctamente a situaciones para las que no se programe una reacción, lo que da lugar a problemas de generalización.
	\item Los conjuntos de datos disponibles contienen información de distintos sensores y cámaras sin procesamiento previo, debido a esto los datos pesan demasiado y no se tiene información organizada y etiquetada.
	
	\item Para lograr una buena autonomía son necesarios componentes de hardware avanzados como un LIDAR que brindan una mejor percepción del espacio, esto ocasiona que el precio de los sistemas de conducción autónoma sea elevado y poco accesible, además de complejizar el problema de implementación.
	
	\item Los modelos de aprendizaje profundo generalizan bien pero un modelo de varias capas es más complejo de entrenar, hace necesario recopilar muchos datos y requiere mucho poder de cómputo.
	
	\item Al entrenar modelos simples de aprendizaje profundo, las representaciones profundas pueden ser subóptimas, esto ocasiona problemas con cambios en el contenido, perspectiva e iluminación.
	
	\item Los algoritmos de visión computacional requieren poca capacidad de cómputo, pero sólo funcionan en condiciones ideales de imágenes para los que se programan, esto da lugar a problemas de generalización de las predicciones.
	
	\item Acceder a un vehículo autónomo para realizar pruebas de rendimiento de un modelo es una tarea complicada de lograr, esto hace costoso el desarrollo de la tecnología y la mejora de los modelos disponibles para esta tarea.
\end{itemize}