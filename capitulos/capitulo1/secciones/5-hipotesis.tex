\section{HIPÓTESIS}
% El flujo permite el entrenamiento de modelos de aprendizaje profundo con altos porcentajes de exactitud, que en conjunto con algoritmos de visión computacional, logran una conducción autónoma básica en situaciones de carreteras y pistas de competencia.
%El uso de aprendizaje profundo y algoritmos de visión computacional, permiten plantear un modelo con el cual un vehículo logra la autonomía básica en caminos de doble vía
El modelo de conducción autónoma mediante el uso de aprendizaje profundo y algoritmos de visión computacional alcanza una autonomía de nivel 2 en vías de doble sentido con separación física.

\subsection{OPERACIONALIZACIÓN DE VARIABLES}

\begin{center}
    \begin{tabular}{|c|c|}
        \hline
        \textbf{Variable} & \textbf{Tipo} \\
        \hline
        Aprendizaje profundo y algoritmos de visión computacional & Independiente \\
        \hline
        Modelo de conducción autónoma & Interviniente\\
        \hline
    \end{tabular}\\
    \vspace{2mm}
    \textbf{Fuente:} Elaboración propia
\end{center}