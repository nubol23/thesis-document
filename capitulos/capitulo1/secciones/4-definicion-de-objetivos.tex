\section{DEFINICIÓN DE OBJETIVOS}
%\section[Definición de Objetivos]{DEFINICIÓN DE OBJETIVOS}

\subsection{OBJETIVO GENERAL}
%\subsection[Objetivo General]{OBJETIVO GENERAL}
% Desarrollar un sistema de conducción autónoma mediante el uso de técnicas de visión computacional clásica y moderna.

% Desarrollar un flujo de entrenamiento e inferencia de algoritmos de visión computacional y modelos de aprendizaje profundo aplicados a las tareas necesarias para conseguir un nivel de autónoma básico en pistas y carreteras.

% Desarrollar un flujo que permita el entrenamiento de modelos de aprendizaje profundo que combinados con algoritmos de visión computacional logren una conducción autónoma básica en pistas y carreteras.
Plantear un modelo para la conducción autónoma que logre una autonomía básica en vías de doble sentido con separación física.

%3 y 5
\subsection{OBJETIVOS ESPECÍFICOS}
%\subsection[Objetivos Específicos]{OBJETIVOS ESPECÍFICOS}
\begin{itemize}[nosep]
%    \item Diseñar un componente de aumentación y preprocesamiento de datos para extraer la información útil de los datasets y fuentes de datos disponibles.
	\item Diseñar un componente de aumentación y preprocesamiento de datos para extraer y crear un dataset con el fin de resolver la tarea.
    % \item Reducir la complejidad de la implementación del flujo mediante el uso de sólamente cámaras.
    \item Reducir la complejidad de implementación del modelo mediante el uso de solamente una cámara.
    
%    \item Analizar y entrenar redes neuronales con una alta exactitud en las predicciones utilizando menos requisitos de cómputo.
    \item Modificar y entrenar redes neuronales con una alta exactitud en las predicciones utilizando menos requisitos de cómputo.
    
    \item Analizar las predicciones de los modelos entrenados para comprobar si las representaciones aprendidas son invariantes a los cambios de perspectiva, iluminación y objetos en la imagen.
    
%    \item Implementar un componente que combine la predicción de algoritmos de visión computacional y modelos de aprendizaje profundo para mejorar la generalización de predicciones.
	\item Combinar las salidas de algoritmos de visión computacional y modelos de aprendizaje profundo para mejorar la generalización de predicciones.
    
    \item Probar el rendimiento del modelo en una simulación, analizando casos de fallas y qué situaciones puede manejar correctamente.
\end{itemize}